\section{Answers to the comments of Reviewer 1}

\paragraph{}
\begin{itemize}
\item This paper proposed a vulnerability analysis of power grid based on graph topology and parallelized the analysis using Apache spark system. The problem is important since power grid is important to a country. The paper mentioned there are only a few works on such vulnerability analysis on power grid. This paper gives interesting exploration in this area. 

The drawback is that the approach is specific to the Lebanese power grid network. A more general approach will be more beneficial to the audience.

~

\answer{The authors concede that a more general approach is certainly more beneficial; however, we still believe there is merit in tackling the specifics related to the Lebanese power grid, especially after Lebanon has withstood decades of wars and suffered from a lack of infrastructure rebuilding. With that said, our approach can generalize to networks of low connectivity such as that found in the Lebanese grid. We have clarified this in the Introduction.}

~

\item The paper removes the five graph nodes which are the five major power plants. This removal seems  ad-hoc and is not formally included in their algorithm. It is better to give more graph property based quantatitive measure to remove nodes. This removal is very specific to the Lebanese network and it is not clear how to transfer it to other networks.

~

\answer{Unfortunately, we received the data from the Lebanese ministry of power and energy in such a way that those five major power plants have already been removed. We just made this comment in Sec. III A, second paragraph. It is not clear to the authors whether we would still have to provide for how to remove those nodes, and we would appreciate it if the reviewer can let us know in case this is still required.}

~


\item The ``parallel efficiency'' is not defined.

~

\answer{We now inserted a definition when this term is first used.}

~


\item It is better to give the runtimes for serial phase and parallel phase, since serial phase limits the maximum speedups when using more threads. 

~

\answer{Actually, the timings shown in the tables correspond to the parallel phase of the algorithm. The sequential part was omitted because it was extremely negligible compared to the parallel part that is of higher order (in contrast to linear running time in steps 7 and 8). We should have made this explicit and have by now inserted this comment in the text. Thank you for point it out.}

~


\item The parallelization proposed by the paper is a simple and direct parallelization approach. The parallelization effectiveness strongly depends on the number of connected components and the balance of the size of each component. In this paper, the authors observe each component is very small and the number of components are huge, which may hide the load balance problem. When the graph evolves or change to another network, it is not clear if this property still remains or not.
I did not see how the partitioning of RDD is done in this paper(only different number of partitions is tried in the paper) and partitioning strategy may have strong impact on the parallelization.

~

\answer{
ln Spark, the number of partitions can either be (1) specified by the Spark execution engine so that to pick the best number of partitions with respect to the number available cores and data size; or (2) can be adjusted (using \texttt{repartition} method) by the programmer according the needs of the application. Nonetheless, in both cases, the partitions are created in a balanced way with respect to size. Moreover, GraphX uses a balanced graph partitioning to allow for the efficient computation by a BSP-like distributed computing framework. Additionnaly, Spark allows for the use of custom partitioning (using \texttt{HashPartitioner} and \texttt{RangePartitioner}), which can be defined by the programmer. In our case, we manually varied the number of partitions to study the scalability of our method, and for each partitioning scheme we made sure that the distribution of the partitions is done so that large strongly connected components are spread among different partitions. 
We added a paragraph in the paper to elaborate more about partitioning.
}
 

~

\item The input graph in the paper is small and a single machine is enough for processing it. The spark system designed mainly for distributed environments brings extra cost when running on shared memory machines. 

~

\answer{We agree that the input graph is small, but what we are offering is a proof of concept that the algorithm scales and its performance improves should the graph get larger (for example, in studying the power grid of a bigger country, or networks of networks connected to the power grid such as telephony and banking system, to name a few examples. Also, whilst running a distributed algorithm such as one written in Spark may induce significantly extra cost when run on shared memory, we believe this tends to be the case when the algorithm introduces contention and cache issues such as false sharing on the shared data structures. This does not apply to our algorithm.}

~

\item In Table II-IV, no unit is specified for the runtime. 

~

\answer{We specified the unit, which is seconds, in Tables II-IV.}

~

\end{itemize}


