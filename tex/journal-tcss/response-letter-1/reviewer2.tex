\section{Answers to the comments of Reviewer 2}
%
%%%%%%%%%%%%%%%%%%%%%%%%%%%%%%%%%%%%%%%%%%%%%%%%%%%%%
%%%%%%%%%%%%%%%%%%%%%%%%%%%%%%%%%%%%%%%%%%%%%%%%%%%%%
%


The following points specifically need to be addressed:

\begin{itemize}
\item The manuscript is not clear in defining the used terms and objectives:
\begin{itemize}
\item In abstract, it is not clear what the authors mean by ``maintains optimal work''.

~

\answer{to do}

~

\item In Introduction, ``to avoid data movement''. 

~

\answer{to do}

~
\end{itemize}

\item In Section II, review of Resilience analysis using Graph Theory is missing some key aspects of analysis including
percolation based methods and structural vulnerability analysis both respect to the connectivity properties and
cascading failure studies including various propagation models on graphs.

~

\answer{to do}

~


\item The use of Spark is not well-motivated and adequately discussed in the Introduction. It is not clear what algorithm or analysis on graph the authors would like to apply to the power grid network that will require the use of Spark. 
Also, the Background Sections lacks literature review on power network analysis using Spark or MapReduce.

~

\answer{to do}

~

\item The cascade model is not clear; is it load based or probabilistic or...? In page 4, one of the mechanism to select victim node is defined to be Cascade Attack. What does that mean? Cascade can be triggered but the process of cascade happens due to interactions among components (e.g., load redistribution) not by external force.

~

\answer{to do}

~

\item The physics of the power and other properties of the power system are not mentioned, discussed or considered in the study. A large body of work is emerging on advantages and limitations of pure graph-based analysis for power systems, which need to be mentioned. 

~

\answer{to do}

~

\item Editorial Comments:
\begin{itemize}
\item Please don't use abbreviations without first defining them. Example: BSP in the abstract. Also, add a reference for that.

~

\answer{We revisited the paper by defining abbreviation first. Also, we added a reference for BSP in the introduction. }

~
\item For attackGraph and all the codes presented in the manuscript,  it's better to include the pseudocode in the algorithm format. The current format is not standard!

~

\answer{We re-wrote all the algorithms by following a standard formatting.}

~
\end{itemize}
\end{itemize}



