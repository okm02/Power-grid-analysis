\section{Answers to the comments of Reviewer 2}
%
%%%%%%%%%%%%%%%%%%%%%%%%%%%%%%%%%%%%%%%%%%%%%%%%%%%%%
%%%%%%%%%%%%%%%%%%%%%%%%%%%%%%%%%%%%%%%%%%%%%%%%%%%%%
%


The following points specifically need to be addressed:

\begin{itemize}
\item The manuscript is not clear in defining the used terms and objectives:
\begin{itemize}
\item In abstract, it is not clear what the authors mean by ``maintains optimal work''.

~

\answer{Thank you for point this out. We have now clarified this in the proof of Prop. 5 as follows: A parallel algorithm is work optimal if it does not run more instructions than the sequential version.}

~

\item In Introduction, ``to avoid data movement''. 

~

\answer{We have ellaborate more on data movement, which implies network communications in case of distributed setting, and garbage collector and memory allocation in case of local setting
}

~
\end{itemize}

\item In Section II, review of Resilience analysis using Graph Theory is missing some key aspects of analysis including percolation based methods and structural vulnerability analysis both respect to the connectivity properties and cascading failure studies including various propagation models on graphs.

~

\answer{We have updated Section II by discussing other aspects of analysis~\cite{Bianconi:2016ka,Callaway:vd,Karrer:2014ep,Radicchi:2015gp} such as percolation based method.}


~


\item The use of Spark is not well-motivated and adequately discussed in the Introduction. It is not clear what algorithm or analysis on graph the authors would like to apply to the power grid network that will require the use of Spark. 
Also, the Background Sections lacks literature review on power network analysis using Spark or MapReduce.

~

\answer{Indeed -- the only algorithm that requires the use of Spark is connected components. By large, this isn't a task parallel algorithm. Spark is used to distribute the data and so in a way it implements a data parallel rather than task parallel algorithm. Our objective has been to use serial graph algorithms on distributed data in SPMD mode (single program, multiple data). Please note that we have now inserted these remarks in the introduction.\newline
Also, we have noted that no work uses spark or MapReduce for power grid analysis beyond the ones mentioned in the article (for example, Jin et al’s, 2010).
}

~

\item The cascade model is not clear; is it load based or probabilistic or...? In page 4, one of the mechanism to select victim node is defined to be Cascade Attack. What does that mean? Cascade can be triggered but the process of cascade happens due to interactions among components (e.g., load redistribution) not by external force.

~

\answer{The cascading scenario is based on recalculated information or more precisely after the identification of the most central node of the initial graph and subsequent to its removal a new graph ensues. The central node of the latter should then be computed and then the process is iterated over. We have updated the text to elaborate more on this.}

~

\item The physics of the power and other properties of the power system are not mentioned, discussed or considered in the study. A large body of work is emerging on advantages and limitations of pure graph-based analysis for power systems, which need to be mentioned. 

~

\answer{We have added extra references~\cite{Hines:2007bt,WangST10} that discuss the physical properties of the network.}

~

\item Editorial Comments:
\begin{itemize}
\item Please don't use abbreviations without first defining them. Example: BSP in the abstract. Also, add a reference for that.

~

\answer{We revisited the paper by defining abbreviation first. Also, we added a reference for BSP in the introduction. }

~
\item For attackGraph and all the codes presented in the manuscript,  it's better to include the pseudocode in the algorithm format. The current format is not standard!

~

\answer{We re-wrote all the algorithms by following a standard formatting.}

~
\end{itemize}
\end{itemize}



