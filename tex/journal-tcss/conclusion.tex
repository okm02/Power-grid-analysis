\section{Conclusion}
\label{conclusion}
Contingency analysis is a security function to assess the ability of a power grid to sustain various combinations of power grid component failures at energy control centers. To date, there exists no such work to examine the power grid resilience in Lebanon, a country which is still reeling under the effect of a brutal civil war, and recently, bearing an additional burden associated with the spillover from the Syrian war. This neighbouring conflict has exposed Lebanon to a number of random terrorist attacks, and caused it to become one of the major hosts of Syrian refugees, in addition to Iraqi and Palestinian refugees in former years. The strains on Lebanon's vital infrastructure and economical resilience have also been affecting the host community itself, where electric power supply is becoming increasingly drained. Our analysis of the resilience of Lebanese power grid captures all of the network down to its finest spatial coordinates. The computational burden associated with the resulting big graph is alleviated using a Spark-based, BSP modeled algorithm that balances the work optimally and incurs linear communication cost and a constant amount of synchronisation among threads. Our analysis exploits a high level of decentralisation in the Lebanese power grid, and reveals a high level of redundancy, thanks to Lebanon's widespread reliance on diesel generators for surviving daily power blackouts. Our analysis also reveals the loss of connectivity in the power grid as a function of failures. This function behaves analogously to the case of the North American grid, as can be seen from \cite{2000Natur.406..378A}, for example. 

In addition to the functional requirements described in this paper, the nonfunctional requirements our contingency analysis software provide for all of the following:
\begin{enumerate}
\item Usability: The targeted audience of the system includes governmental and other vital organisations. Our tool is intended to assist governments in mitigating any possible exploitation of the network and the networks that are incumbent on it, for example, the road network, commercial network, telephony.
\item Portability: The system can run on any Hadoop configured machine, which can be a serial machine, a multi-core, or a cluster of machines.
\item Availability: The system is an open source project and can be downloaded from the developer's Github repository at \url{https://github.com/okm02/power-grid-analysis}.
\item Capacity: The capacity of the system adapts to the size of the graph. No further amendments are required, provided the input graph decomposes into a set of connected components each of which can be processed locally on a serial computer. 
\end{enumerate}
