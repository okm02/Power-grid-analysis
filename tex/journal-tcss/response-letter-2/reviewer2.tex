\section{Answers to the comments of Reviewer 2}
%
%%%%%%%%%%%%%%%%%%%%%%%%%%%%%%%%%%%%%%%%%%%%%%%%%%%%%
%%%%%%%%%%%%%%%%%%%%%%%%%%%%%%%%%%%%%%%%%%%%%%%%%%%%%
%


\begin{itemize}
\item The paper has improved by addressing most of the previous comments. However, the presented arguments are not yet convincing as why Spark platform is necessary in the presented analysis. Spark is in the title of the manuscript so this suggests the importance of the role of Spark in the analysis but based on the rest of the paper the use of Spark is not well-motivated and justified. Merely having the capability to use Spark to show that the analysis can be extend is not enough as the presented analysis can be done without Spark in a single computer. I would suggest either tone down the role of Spark or better justify its importance. The importance aspect of this study is the vulnerability analysis and not on what platform it was done, unless one couldn't do this without Spark. As such, the contribution and main focus of the paper needs to be clarified.

~

\answer{
We believe that the use of Spark is necessary to handle large graphs (i.e., scalability), which is not possible to in case of single machine. Using Message Passing Interface (MPI) would be also an option that allows for distributed computation and hence scalability. Nonetheless, when the graph increases failures (because of out of memory, message loss, network error) would appear very often (and not an exception), which requires to redo the whole computation.  Using Spark allows to benefit from its fault-tolerance using lineage technique as well as efficient distributed in-memory computation.  Scalability as well as fault-tolerance are discussed in the paper. We also agree with the reviewer that we should tone down the role of Spark especially in the title. 
}
\end{itemize}



