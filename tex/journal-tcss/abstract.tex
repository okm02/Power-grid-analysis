\begin{abstract}
We address a topological vulnerability analysis of the Lebanese power grid subject to random and cascading failures. Using an Apache Spark implementation that maps the topology of the grid to a complex network, we begin by developing a local structural understanding of the Lebanese power grid that reveals a certain level of decentralisation via numerous connected components. Our Apache Spark implementation simulates random and cascading sequences of events by which energy centers in Lebanon can be exposed and are at risk. The implementation is based on the bulk-synchronous parallel (BSP) model and maintains optimal work, linear communication time, and a constant number of synchronisation barriers. We complement our work with a spatial understanding of the exposed hotspots. Our results reveal that failures in the power grid are spatially long-range correlated, and that correlations decay with distance. In a couple of attack scenarios our Spark implementation achieves significant speedup on $16$ cores for a graph with about $9\times 10^5$ nodes. Scalability towards $32$ nodes improves when experimenting with replicas of the power grid graph that are double and quadruple the original size. This renders our work suitable to larger networks at many vital levels beyond the power grid.
\end{abstract}
