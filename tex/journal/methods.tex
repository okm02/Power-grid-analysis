\section{Materials and Methods}
\label{methods}
%We begin by developing a global structural understanding of the Lebanese power grid that reveals a certain level of decentralization via numerous strongly connected components, which, surprisingly, may be perceived as a sign of modest resilience against overload or attacks. 
We  simulate the failures by node removal sorted in a random or a particular order given by the nodes with the highest betweenness, nodes with the highest betweenness in cascading order resulting from their dynamical removal, the nodes with the highest loads. 
%We develop an Apache Spark implementation that simulates random and cascading sequences of events by which energy centers in Lebanon can be exposed and are at risk. Spark is an Apache project currently dubbed as lightning fast cluster computing, and allows to deploy jobs on distributed and parallel systems \cite{spark}. It has become the fastest open source engine for performing major computations like sorting, graph algorithms, and machine learning, on data of the order of petabytes. 

Given an undirected graph $G=(V,E)$, we define its connectivity as follows:

$$
\mathtt{connectivity}(G) = \sum_{v \in V} \mid \mathtt{reachable}(v) \mid,
$$ where $\mathtt{reachable}(v)$ is all the reachable nodes from $v$.

Given a node $x$, we define the loss of a graph $G$ with respect to $x$ as follows:
$$
\mathtt{loss}(G, x) = \mathtt{connectivity}(G) - \mathtt{connectivity}(G \setminus x)
$$
where $G \setminus x$ is a graph defined by removing vertex $x$ in $G$ and all its edges. 

We denote by $\mathtt{SCC}(G) = \{G_1, \ldots, G_n\}$ to be the maximal strongly connected components of $G$. 
We have $\mathtt{loss}(G, x) = \mathtt{loss}(G_i,x) + \sum_{j \neq i} \mathtt{connectivity}(G_j)$, where $x \in V_i$, and  $\mathtt{connectivity}(G_j) \,=\, \mid V_j \mid \times \mid V_j  - 1\mid$.

\begin{lstlisting}[language=java]
attackGraph(Graph graph) {
   for(i = 0 until |V|) {
      select victim vertex v
      remove vertex v
      update loss with respect to v
   }
}
\end{lstlisting}

We select a victim vertex according to one of the following four scenarios: 
\begin{itemize}
\item \emph{Random}: a random vertex is selected.
\item \emph{Degree-based}: the vertex with the highest degree is selected. 
\item \emph{Betweenness Centrality}: the vertex with the highest betweeness centrality is selected. The betweenss centrality of the nodes is only computed once (i.e., it will not be updated after removing a vertex). 
\item \emph{Cascanding}: Similar to the betweeness centrality scenario, however, the betweeness centrality of the nodes is updated at each iteration (i.e., after removal of a victim vertex).
\end{itemize}

Given a graph $G$, the betweeness centrality of a node $v$ is equal to $\mathtt{bc}_G(v) = \sum_{s \neq v \neq t} \frac{\sigma_{st}(v)}{\sigma_{st}}$, where  
$\sigma_{st}$ is the total number of shortest paths from node 
$s$ to node $t$ and $\sigma_{st}(v)$ is the number of those paths that pass through $v$. Clearly, we have 
$\mathtt{bc}_G(v) = \mathtt{bc}_{G_i}(v)$, where $\mathtt{SCC}(G) = \{G_1, \ldots, G_n\}$  and $v \in G_i$. 


\subsection{Spark-based Implementation}
We provide an efficient Spark-based implementation of the all the above scenarios. %Spark allows to parallelize and distributed computations on several nodes. 
%
%Spark~\cite{spark} is new programming model supported by an execution engine for big data processing. Spark is based on RDD (Resilient Distributed Data Set). RDDs are big parallel collections that can be distributed across a cluster. RDDs are created through parallel transformations (e.g., map, group by, filter, join, create from file systems). Moreover, RDDs can be cached (in-memory) by allowing to keep data sets of interest in Memory across operations and thus contribute to a substantial speedup. At the same time, Spark uses lineage to support fault tolerance, i.e., record all the operations/transformations that yield to create RDDs from a source data. That is, in case of failure, an RDD can be reconstructed given the transformation functions yielding to that RDD. Additionally, after creating RDDs, it is possible to do analytics on them by running actions on them such as count, reduce, collect and save. Note that all operations/transformations are lazy until you run an action. Then, the Spark execution engine pipelines operations and finds an execution plan. GraphX~\cite{graphx} is a platform built on top of Spark that provides APIs for parallel and distributed processing on large graphs. 
%
The spark-based implementation is done as follows:
\begin{itemize}
\item Given a file (read from local or distributed file system, i.e., HDFS) containing information about the graph and a number of partitions, we build the corresponding graph RDD. 
\item We compute the strongly connected components on the graph RDD. 
\item We create an RDD, \texttt{rddSCC} where each item corresponds to a strongly connected component. 
\item On each item \texttt{rddSCC} (i.e., each strongly connected component), we iteratively identifies victim nodes (and locally update the corresponding component) until all the nodes are identified. The output of each step produces an array per item containing the losses in a decreasing order that correspond to the nodes of that item. Note that, the loss is computed and affected locally on each item. 
\item We define a reduce operation that merge-sorts all the generated arrays and sequentially removes the nodes in a decreasing order. The output of this step produces one array containing the global losses (in a decreasing order) corresponding to all the nodes. Global losses are computed by by accumulating the sum of all the previous items. 
\end{itemize}




