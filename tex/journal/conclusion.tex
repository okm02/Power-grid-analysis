\section{Conclusion}
\label{conclusion}
Contingency analysis is a security function to assess the ability of a power grid to sustain various combinations of power grid component failures at energy control centers. To date, there exists no such work to examine the power grid resilience in Lebanon, a country which is still reeling under the effect of a brutal civil war, and recently, bearing an additional burden associated with the spillover from the Syrian war. This neighbouring conflict has exposed Lebanon to a number of random terrorist attacks, and caused it to become one of the major hosts of Syrian refugees, in addition to Iraqi and Palestinian refugees in former years. The strains on Lebanon's vital infrastructure have also been affecting the host community itself, where electric power supply is becoming increasingly drained. Coupled with a severe shortage of funds that can help restructure the entire system, the Lebanese power grid has become plagued with nationwide blackouts, making it an unreliable source of electricity generation, transmission, and distribution. 

In addition to the functional requirements described in this paper, the nonfunctional requirements our contingency analysis software provide for all of the following:
\begin{enumerate}
\item Usability: The targeted audience of the system includes governmental and other vital organisations. Our tool helps governments expose any network to failures or destruction of it components and thus identifying the network��s resistance to such attack. Helping governments in preparing necessary measures to mitigate any possible exploitation of the network. Such networks include: road network, commercial network, telephony, ...etc.
\item Portability: The system can run on any Hadoop configured machine, which can be local, serial machines all the way to cluster computing machines.
\item Availability: The system is an open source project and can be downloaded from the developer��s Github repository at \url{?}
\item Capacity: The capacity of the system adapts to the size of the graph. No further amendments necessary at all, provided the input graph decomposes into a set of connected components each of which can be processed locally on a serial computer. 
\end{enumerate}
