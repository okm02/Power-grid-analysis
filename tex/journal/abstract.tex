\begin{abstract}
We address a topological vulnerability analysis of the Lebanese power grid subject to random and cascading failures. The seminal precursor to our work follows from \cite{2000Natur.406..378A} in mapping the topology of complex networks. Using an Apache Spark implementation suitable for big datasets, we adapt this body of work to the peculiarities of the Lebanese power grid. We begin by developing a global structural understanding of the Lebanese power grid that reveals a certain level of decentralisation via numerous strongly connected components. Our Apache Spark implementation that simulates random and cascading sequences of events by which energy centers in Lebanon can be exposed and are at risk. We conclude with a spatial understanding of the hotspots on Lebanese soil where energy centers can be exposed and are at risk, using a spatial correlation kernel supported by a binary search tree. Our Spark implmentation achieves significant speedup on ? cores and concludes in real time for a graph with about $800,000$ nodes. The amenability of our work to big data processing makes it extendable to larger networks of networks, towards a fuller understanding of resilence at many vital levels beyond the power grid.
\end{abstract}
