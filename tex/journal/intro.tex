\section{Introduction}
In scale-free networks (SF) the probability of a node being connected to $k$ others exhibits a power-law distribution $P(k) \propto k^{-\alpha}$, which is a topological property affecting and controlling their resilience or the measure of their functionality subject to disruptions \cite{Newman:2003da,Gao:2015fg,Bashan:2013cja,Gao:2015fga,2000Natur.406..378A}. Examples of this class of SF are the Internet, power systems, and transportations networks, which are real-world networks shown to be robust when subject to random failures however display a high vulnerability when prone to cascading ones \cite{Bompard:2011cd,DuenasOsorio:2009ff,2016arXiv160904310M,Cohen:2001hf}. 

In power systems, and unlike random failures which emerge locally, blackouts are severe events  associated with cascading behavior leading to global network collapse \cite{RosasCasals:2007td,Bompard:2009ga,Brummitt:2013jj,Daqing:2014bp,Albert:2004bw,Wang:2011js,Sole:2008cv}.
Examples such as the one affecting the north-east in the US and eastern Canada in 2003 burdened their economies with 10 billions dollars of direct costs \cite{Daqing:2014bp}. Such failures can be linked to either structural dependencies, where the damage spreads via structurally dependent connections, or functional overloads, where the flow goes through alternative paths leading to overloaded nodes. Thus understanding the propagation of these failures becomes pivotal in developing and deploying protective and mitigating strategies. 

In Lebanon the power grid is plagued with nationwide blackouts making it an unreliable source of electricity generation, transmission, and distribution. In this paper we investigate the reliability and robustness of the power grid in four scenarios where failures affect: the nodes with highest betweenness, the nodes with the highest betweenness in cascading order resulting from their dynamic removal, the nodes with the highest loads, and random ones. In addition, we explore the spatial correlation between the failures in the first three cases. 
